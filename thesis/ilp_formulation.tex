\section{Final Considerations}

This study focuses on the development and evaluation of an algorithm specifically designed to detect inversions in genomic sequences obtained from long reads. The experimental results show that the algorithm performs efficiently for moderate reference lengths and inversion counts. However, its execution time increases significantly as the reference sequence length grows. This behavior is consistent with the anticipated time complexity of \( \theta(r + n \cdot m) \), where the reference length \( r \) is the dominant factor. A significant advantage of the algorithm is its ability to effectively handle inverted duplications, expanding the method's applicability to real-world genomic data where such variations are common. \\
Overall, the algorithm provides a reliable approach for inversion detection, although further optimizations may be necessary to manage even larger datasets or more complex genomic structures effectively. It is also essential to note that the algorithm operates under the assumption that the input samples are error-free. Any presence of sequencing errors or mutations leads to incorrect identification of inversions, ultimately affecting the overall performance and accuracy of the algorithm.  

\section{Future improvements}

Future improvements to the inversion detection algorithm will focus on two key areas: enhancing its capability to handle sequencing errors and mutations, and increasing its efficiency for processing long reference sequences that may span millions of base pairs. Given that it is almost impossible to obtain perfectly clean reads in real-life scenarios, the current implementation of the algorithm may not be advisable for practical use. Additionally, experimental results demonstrate that as the reference length increases, the execution time of the algorithm rises significantly, highlighting the need for optimization in this area. \\
Addressing these areas will significantly enhance the algorithm's reliability and applicability in genomic research, particularly when dealing with extensive datasets typical of modern genomic studies.  